\documentclass[11pt]{beamer}
\usetheme{Warsaw}
\usepackage[utf8]{inputenc}
\usepackage{amsmath}
\usepackage{amsfonts}
\usepackage{amssymb}
\author{Beau Horenberger and Kirk Bonney}
\title{Calculating Coefficients for Fourier Expansions of Siegel Modular Forms}
%\setbeamercovered{transparent} 
%\setbeamertemplate{navigation symbols}{} 
%\logo{} 
\institute{University of Idaho} 
%\date{} 
%\subject{} 
\begin{document}

\begin{frame}
\titlepage
\end{frame}

\begin{frame}
\frametitle{Plan of Attack}
\tableofcontents
\end{frame}

\section{Modular Forms}
\subsection{Modular Forms and Periodicity}
\begin{frame}{•}
\frametitle{Modular Forms}
\begin{itemize}
\item Generalize periodicity $f(x+kp)=f(x)$, where $p$ is the period, $k\in \mathbb{R}$
\item \textbf{Modular forms} are functions on complex numbers with a positive imaginary part. 
\item $\forall z \in H$ and for all matrices in $SL(2, \mathbb{Z})$, modular forms satisfy,
	$$f\left( \frac{\alpha z + \beta}{\gamma z + \delta} \right) = (cz+d)^k f(z)$$
\item A convenient consequence: the periodicity allows for Fourier Expansions!
\end{itemize}
\end{frame}

\begin{frame}{•}
\frametitle{Fourier Expansions for Modular Forms}
\begin{itemize}
\item What does a Fourier expansion look like? $$f(\tau) = \sum_{n=0}c(n)e^{2\pi i n \tau}$$
\item This is just sine and cosine waves added together
\item Our object of interest is $c(n)$.
\begin{itemize}
\item They reveal a lot about how $f(\tau)$ "looks"
\item We have lots of these documented on websites like LMFDB
\end{itemize}
\end{itemize}
\end{frame}

\subsection{Generalizing Modular Forms}
\begin{frame}{•}
\frametitle{Generalizing Modular Forms}
\begin{itemize}
\item \textbf{Siegel modular forms} are functions on complex symmetric matrices with positive imaginary parts
\item It also satisfies some behavior, but where modular forms are periodic on a subset of $SL(2, \mathbb{Z})$, Siegel modular forms are periodic on a subset of the \textbf{symplectic group}:
\end{itemize}
$\Gamma_{g}(N)=\left\{\gamma \in G L_{2 g}(\mathbb{Z}) | \gamma^{\mathrm{T}} \left( \begin{array}{cc}{0} & {I_{g}} \\ {-I_{g}} & {0}\end{array}\right) \gamma=\left( \begin{array}{cc}{0} & {I_{g}} \\ {-I_{g}} & {0}\end{array}\right), \gamma \equiv I_{2 g} \quad \bmod N\right\}$
\begin{itemize}
\item We call $g$ the \textbf{genus}
\item This sets us up to describe periodicity on Siegel modular forms
\end{itemize}
\end{frame}

\section{Siegel Modular Forms and their Coefficients}
\subsection{Properties of Siegel Modular Forms}
\begin{frame}
\frametitle{Convenient transition}
\begin{itemize}
\item Our work focuses on Siegel paramodular forms
\begin{itemize}
\item This just means a Siegel modular form of genus 2.
\end{itemize}
\item Since we have periodicity on some subset of $\Gamma_{g}(N)$ (trust us, we do), we have a Fourier expansion for our Siegel paramodular forms
\item And now, we are interested in learning about these expansions and their Fourier coefficients
\end{itemize}
\end{frame}

\subsection{Binary Quadratic Forms and Fourier Expansions}
\begin{frame}
\frametitle{Coefficients and Binary Quadratic Forms}
\begin{itemize}
\item Our new Fourier expansion looks like: $$F(Z) = \sum_{S\in A(N)^+}a(S)e^{2\pi itr(SZ)}$$
\item Our coefficients, $a(S)$, are a different this time: they're indexed on $\left( \begin{array}{cc}{a} & {b} \\ {b} & {d}\end{array}\right) \in A(N)^+$, matrices where \\ $N\vert a$, $b\in \frac{1}{2}\mathbb{Z}$, $a > 0$, and $ac-b^2 > 0$
\item It is common to think of these as \textbf{binary quadratic forms}, equations of the form $ax^2+bxy+cy^2$
\end{itemize}
\end{frame}
\begin{frame}
\frametitle{Calculating New Coefficients}
\begin{itemize}
\item We have: sets of coefficients $a(n)$ and their respective indexes $\left( \begin{array}{cc}{a} & {b} \\ {b} & {d}\end{array}\right)\in A(N)^+$
\item We have: five formulas which produce new coefficients for subsets of our current coefficients. Example 1: $$a_{\chi}(S)=p^{1-k} \chi(2 \beta) \sum_{b \in(\mathbb{Z} / p \mathbb{Z})^{ \times}} \chi(b) a\left( S[\left[ \begin{array}{cc}{1} & {-b p^{-1}} \\ {} & {p}\end{array}\right]]\right)$$
\end{itemize}
\end{frame}

\section{Our Work}
\subsection{Divisibility and Binary Quadratic Forms}
\begin{frame}
\frametitle{Determining Which Coefficients Have Solutions}
\begin{itemize}
\item There are many matrix operations occurring here
\item We know our matrices have integer values
\item It turned out a minority of observed samples have solutions for our example!
\end{itemize}
\end{frame}

\begin{frame}
\frametitle{An Example Claim}
Claim:
\\
Let $S$ be a binary quadratic form such that $S=
\begin{bmatrix}
x &	y \\
y & z
\end{bmatrix}$ with discriminant $D$ 
such that $p^4\vert x$ and $p\nmid 2(bx+py)/p^2$, $$S'=(S[
\begin{bmatrix}
1 &	-bp^{-1} \\
0 & p
\end{bmatrix}
]^{-1})=\begin{bmatrix}
x &	(bx +py)/p^2 \\
(bx+py)/p^2 & (b^2x+2bpy+p^2z)/p^4
\end{bmatrix}$$. Then $S'$ is integer-valued $\implies$ $p^2\vert\vert D$ and $p\nmid z$

\end{frame}

\subsection{Equivalence and Higher Levels}
\begin{frame}
\frametitle{Equivalence and Generally Making Things Harder}
\begin{itemize}
\item Not all Siegel modular forms are so easy!
\item Once we know which coefficients map to new coefficients, we must also be sure they are unique
\item This depends on the \textbf{level} of the Siegel modular form, but the problem is analogous to the number theory problem of equivalence of binary quadratic forms
\item For more information, see David Cox's "Primes of the form $x^2+ny^2$"
\end{itemize}
\end{frame}

\section{Conclusion}
\begin{frame}
\frametitle{Thanks for your patience!}
Thank you for your time! We hope you enjoyed!
\end{frame}

\end{document}